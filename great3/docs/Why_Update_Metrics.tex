\documentclass[preprint,11pt]{aastex}

% Copyright 2012, 2013 The GalSim developers:
% https://github.com/GalSim-developers
%
% This file is part of GalSim: The modular galaxy image simulation toolkit.
%
% GalSim is free software: you can redistribute it and/or modify
% it under the terms of the GNU General Public License as published by
% the Free Software Foundation, either version 3 of the License, or
% (at your option) any later version.
%
% GalSim is distributed in the hope that it will be useful,
% but WITHOUT ANY WARRANTY; without even the implied warranty of
% MERCHANTABILITY or FITNESS FOR A PARTICULAR PURPOSE.  See the
% GNU General Public License for more details.
%
% You should have received a copy of the GNU General Public License
% along with GalSim.  If not, see <http://www.gnu.org/licenses/>
%

% packages for figures
\usepackage{graphicx,times}
% packages for symbols
\usepackage{latexsym,amssymb,hyperref}
% AMS-LaTeX package for e.g. subequations
\usepackage{amsmath}

%=====================================================================
% FRONT MATTER
%=====================================================================

\slugcomment{\today}

%=====================================================================
% BEGIN DOCUMENT
%=====================================================================

\begin{document}

\setlength{\parskip}{2.0ex plus 0.5ex minus 0.5ex}
\setlength{\parindent}{0cm} 

\title{Why Have We Updated the Metrics for GREAT3?}

You may have noticed that your metric scores in the GREAT3
Leaderboards have changed slightly.  These changes will be most
noticeable for the highest scoring entries, but \emph{all} scores have
changed by some amount.  This document will explain why these change
were made, and give the definitions of the new metrics.

\section{Background}
As mentioned in the first versions of the GREAT3 Handbook (Mandelbaum
et al 2013), we always felt it
might be necessary to update the metrics as we received more
information about methods in the form of entries to GREAT3.  In
particular we felt it might be necessary to \emph{renormalize} the
metrics, to make it so that a method scores $\sim 1000$ if it reaches
target performance.

So why would either the constant shear metric ($Q_{\rm c}$) or
variable shear metric ($Q_{\rm v}$) need renormalization?

To answer this question, let us consider a simple model of the
contributions to an observation of a (complex) gravitational shear $g_{\rm obs}$.
We may write this as a combination of the gravitational reduced shear $g$, galaxy intrinsic ellipticity
$g_{\rm int}$, and a measurement noise term $g_{\rm noise}$ due to the
pixel noise in galaxy images:
\begin{equation}
g_{\rm obs} \simeq g + g_{\rm int} + g_{\rm noise}.
\end{equation}
As they are both noise terms as far as shear measurement is concerned,
it is not uncommon in weak lensing to refer to $g_{\rm int}$ as \emph{shape
  noise} and $g_{\rm noise}$ as \emph{measurement noise} (perhaps this sounds
better than `noise noise'...).

In GREAT3 we have designed our experiments with inherent symmetries
that allow us to cancel out the contributions from intrinsic ellipticity
$g_{\rm int}$ to uncertainty on shear (see Mandelbaum et al 2013).
This allows us to reduce the  total data volume of the challenge
significantly.  However, we have not devised a means to cancel out
contributions from $g_{\rm noise}$, and it is this noise which
ultimately sets data volume required to reach a given sensitivity on
shear biases in methods.

When simulating the performance of various metrics -- for many different
$Q_{\rm c}$ and $Q_{\rm v}$ have been tested using simulated GREAT3
submissions with a range of biases -- we modelled the measurement
noise as Normally distributed with zero mean and standard deviation
set by the variable \texttt{NOISE\_SIGMA}:
\begin{equation}
g_{\rm noise} \sim \mathcal{N}(0, \texttt{NOISE\_SIGMA}^2).
\end{equation}
For all the simulations that went into testing and normalizing the
metrics in the GREAT3 Handbook, a value of \texttt{NOISE\_SIGMA} =
0.05 was estimated as being appropriate to our images.

This figure was based on the decision to include only galaxy images
with signal-to-noise ratio (SNR) $\gtrsim 20$ in GREAT3, using the SNR
defined by Bridle et al (2008).  Assuming a steep distribution of
galaxy SNRs so that the population is dominated by objects with ${\rm
  SNR} \sim 20$, and invoking the approximate `rule of thumb' $g_{\rm
  noise} \sim ({\rm SNR})^{-1}$, gave rise to the estimate
$\texttt{NOISE\_SIGMA} \simeq 0.05$ used in the simulations of metrics
and the calculation of an appropriate normalization factor for $Q_{\rm
  v}$.


\section{Galaxy image signal-to-noise}\label{sec:snr}


\section{The Constant Shear Metric $Q_{\rm c}$}\label{sec:qc}

\section{The Variable Shear Metric $Q_{\rm v}$}\label{sec:qv}

\end{document}
